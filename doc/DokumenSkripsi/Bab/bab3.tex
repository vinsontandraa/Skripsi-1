\chapter{Percobaan Awal}
\label{chap:percobaan_awal}
Pada bab ini akan dijelaskan analisis masalah penelitian ini. Analisis meliputi Eksplorasi Teknologi, Dataset Pada HTTP Archive, Langkah-Langkah Query Yang Dilakukan, dan Hasil Sample Data Dengan Beberapa Aplikasi.

\section{Eksplorasi Teknologi}
Dalam pengerjaan skripsi ini akan menggunakan teknologi bernama BigQuery. Dataset pada HTTP Archive didapatkan dengan menggunakan teknologi BigQuery.
Berikut ini adalah langkah untuk mendapatkan dataset tersebut:
\begin{enumerate}
	\item Membuka Google Cloud Project Page\footnote{https://console.cloud.google.com/getting-started} dan masuk dengan menggunakan Google account.
	\begin{figure}[H]
		\centering  
		\includegraphics[scale=0.35]{Gambar/Google BigQuery HomePage.PNG}  
		\caption{Halaman Awal Google Cloud Project} 
		\label{fig:GCP_HP} 
	\end{figure}
	
	\item Memilih project kemudian \textit{"New Project"}
	\begin{figure}[H]
		\centering  
		\includegraphics[scale=0.35]{Gambar/create_new_project.PNG}  
		\caption{Memilih \textit{Project}} 
		\label{fig:select_project} 
	\end{figure}
	
	\item Masukkan nama \textit{project} kemudian tekan tombol \textit{create}
	\begin{figure}[H]
		\centering  
		\includegraphics[scale=0.45]{Gambar/create_project.PNG}  
		\caption{Membuat \textit{Project}} 
		\label{fig:create_project} 
	\end{figure}
	
	\item Buka BigQuery \textit{console} 
	\begin{figure}[H]
		\centering  
		\includegraphics[scale=0.35]{Gambar/bq_workspace.PNG}  
		\caption{Membuka \textit{Console}} 
		\label{fig:create_project} 
	\end{figure}
	
	\item Untuk menambahkan tabel HTTP Archive pada \textit{project} didapatkan dari link \footnote{https://console.cloud.google.com/bigquery?p=httparchive}
	
	
	\item Data HTTP Archive dapat dilihat pada dashboard BigQuery.
	\begin{figure}[H]
		\centering  
		\includegraphics[scale=0.35]{Gambar/BG-Dashboard.PNG}  
		\caption{Data Terlihat Pada Dashboard} 
		\label{fig:BQ-Dashboard} 
	\end{figure}
\end{enumerate}

Di dalam BigQuery, terdapat salah satu fitur yang akan digunakan yaitu membuat dataset baru. Dataset bisa saja diambil dari public dataset maupun membuat sendiri dataset tersbut. Dataset berisi tabel-tabel yang akan dianalisis. Tabel-tabel tersebut dapat dibuat secara manual maupun di-\textit{upload}.

Berikut ini langkah-langkah dalam pembuatan dataset dan tabel:
\begin{enumerate}
	\item Membuka Google Cloud Project Page\footnote{https://console.cloud.google.com/getting-started}. Halaman yang akan ditampilkan dapat dilihat pada gambar \ref{fig:GCP}
	\begin{figure}[H]
		\centering  
		\includegraphics[scale=0.45]{Gambar/open_GCP.PNG}  
		\caption{Google Cloud Project Page} 
		\label{fig:GCP} 
	\end{figure}
	\item Membuat atau memilih \textit{project} yang akan dikerjakan. Halaman yang akan ditampilkan dapat dilihat pada gambar \ref{fig:create_or_open}
	\begin{figure}[H]
		\centering  
		\includegraphics[scale=0.45]{Gambar/pilih_project.PNG}  
		\caption{Create atau Open Project} 
		\label{fig:create_or_open} 
	\end{figure}
	\item Membuka \textit{console} kemudian memilih BigQuery. Halaman yang akan ditampilkan dapat dilihat pada gambar \ref{fig:BQ}
	\begin{figure}[H]
		\centering  
		\includegraphics[scale=0.45]{Gambar/console_BigQuery.PNG}  
		\caption{Membuka BigQuery} 
		\label{fig:BQ} 
	\end{figure}
	\item Pada tab explorer terdapat project kemudian pengguna harus menekan tombol titik tiga dan piliih \textit{create} dataset. Halaman yang akan ditampilkan dapat dilihat pada gambar \ref{fig:create_dataset}
	\begin{figure}[H]
		\centering  
		\includegraphics[scale=0.45]{Gambar/create_dataset.PNG}  
		\caption{Membuat Dataset Baru} 
		\label{fig:create_dataset} 
	\end{figure}
	\item Buka dataset, kemudian pilih menu \textit{create table}. Halaman yang akan ditampilkan dapat dilihat pada gambar \ref{fig:create_table}
	\begin{figure}[H]
		\centering  
		\includegraphics[scale=0.45]{Gambar/create_table.PNG}  
		\caption{Membuat Tabel Baru} 
		\label{fig:create_table} 
	\end{figure}
\end{enumerate}

Skripsi ini akan membuat tabel baru agar tidak melakukan query yang sama berulang. Tabel dapat dibuat dengan cara:
\begin{enumerate}
	\item Membuat query yang akan disimpan dalam tabel
	\begin{figure}[H]
	\centering  
	\includegraphics[scale=0.35]{Gambar/membuat query.PNG}  
	\caption{Membuat Tabel Baru} 
	\label{fig:create_table} 
\end{figure}
	\item Memilih save result as BigQuery Table
	\begin{figure}[H]
		\centering  
		\includegraphics[scale=0.35]{Gambar/save bigquery table.PNG}  
		\caption{Memilih Save Result As BigQuery Table} 
		\label{fig:save_table}
	\end{figure}
	\item Memilih lokasi atau dataset dan nama tabel untuk disimpan kemudian export
	\begin{figure}[H]
		\centering  
		\includegraphics[scale=0.35]{Gambar/export table.PNG}  
		\caption{Export Table} 
		\label{fig:export_table}
	\end{figure}
	\item Lokasi tabel dapat dilihat pada dashboard
		\begin{figure}[H]
		\centering  
		\includegraphics[scale=0.35]{Gambar/saved dashboard.PNG}  
		\caption{Dashboard Table} 
		\label{fig:export_table}
	\end{figure}
\end{enumerate}

\section{Dataset yang Digunakan pada HTTP Archive}
Dataset pada HTTP Archive yang digunakan pada skripsi ini adalah sebagai berikut:
\begin{enumerate}
	\item technologies
	Pada tabel technologies terdapat beberapa kolom seperti url, category, app, dan info. Url adalah alamat dari sebuah website. Contoh dari dataset dapat dilihat pada tabel \ref{table:ct_tech_desktop} 
	\begin{table}[H]
		\centering
		\begin{tabular}{|l|l|p{3cm}|p{3cm}|l|}
			\hline
			\textbf{Row} & \textbf{url} & \textbf{category} & app & info\\
			\hline
			1 & https://www.3-king.com/ & Analytics & Google Analytics & \\
			\hline
			2 & https://www.fleabites.net/ & Miscellaneous & Twitter Emoji (Twemoji) & \\
			\hline
			3 & http://www.elcarnicero.cl/ & Widgets & OWL Carousel & \\
			\hline
			4 & https://thankyou.ws/ & Analytics & Google Analytics & \\
			\hline
			5 & https://rogerwaters.com/ & Reverse proxies & Nginx & \\
			\hline
			6 & http://www.palaciodaslampadas.com.br/ & JavaScript libraries & jQuery & 2.1.1\\
			\hline
			7 & https://copenhagencamping.dk/ & CMS & WordPress & \\
			\hline
			8 & https://eachat.ma/ & Ecommerce & WooCommerce & 4.3.0\\
			\hline
			9 & https://advokat-bondarchuk.ru/ & Blogs & WordPress & \\
			\hline
			10 & https://passport.rsl.ru/ & JavaScript libraries & jQuery & 1.7.1\\
			\hline
		\end{tabular}
		\caption{Technologies Desktop Data Sample}
		\label{table:ct_tech_desktop}
	\end{table}
\end{enumerate}

\section{Langkah-Langkah Query Yang Dilakukan}
\label{langkah_query}
Pada section ini akan dijelaskan tentang langkah-langkah \textit{query} yang dilakukan dalam memperoleh data. Data yang diambil adalah data percobaan sebanyak 10 data. Data yang diambil merupakan dataset dari tabel technologies 2020$\_$08$\_$01:

\subsection{Mengumpulkan Daftar Website}
Langkah pertama yang dilakukan yaitu mengumpulkan website. Website yang dicari tidak berdasarkan \textit{rank} karena tidak tersedia pada dataset tersebut. Berikut adalah \textit{query} yang digunakan untuk mengumpulkan daftar website.
\begin{lstlisting}
SELECT url
FROM `httparchive.technologies.2020_08_01_*`
GROUP BY url
LIMIT 10
\end{lstlisting}

Pada \textit{query} diatas akan dilakukan pemilihan pada kolom url dengan menggunakan perintah SELECT dari project httparchive dataset \textit{technologies} tabel 2020\_08\_01\_* dengan menggunakan perintah FROM. Mengelompokan pada kolom url yang dilakukan dengan menggunakan perintah GROUP BY sehingga tidak ada nama url yang sama. Kolom akan dibatasi sebanyak 10 baris dengan menggunakan perintah LIMIT. Sepuluh contoh hasil keluaran dari \textit{query} diatas dapat dilihat pada \ref{table:contoh_langkah1}:

\begin{table}[H]
\centering
\begin{tabular}{|l|l|}
	\hline
	\textbf{Row} & \textbf{url}\\
	\hline
	1 & https://www.theinsider.life/\\
	\hline
	2 & http://www.mtctutorials.com/ \\
	\hline
	3 & https://noticias24horases.com.br/\\
	\hline
	4 & https://www.tonyburke.com.au/ \\
	\hline
	5 & http://www.bakedbyjoanna.com/\\
	\hline
	6 & https://stuftburgerbar.com/\\
	\hline
	7 & https://www.skagitpowersports.com/\\
	\hline
	8 & http://www.arazatimaderas.com/ \\
	\hline
	9 & https://oasisexc.com/\\
	\hline
	10 & https://www.captainslanding.com/\\
	\hline
\end{tabular}
\caption{Hasil Pengumpulan Daftar Website}
\label{table:contoh_langkah1}
\end{table}

\subsection{Mencari Aplikasi Yang Digunakan Website}
Setiap website akan dicari aplikasi apa saja yang digunakan dalam pembangunan website tersebut dari aplikasi yang dipakainya. Berikut adalah \textit{query} yang digunakan.
\begin{lstlisting}
SELECT DISTINCT url, app
FROM `httparchive.technologies.2020_08_01_*`
ORDER BY url asc
LIMIT 10
\end{lstlisting}

Pada \textit{query} diatas akan dilakukan pemilihan pada kolom url dan app dengan menggunakan perintah SELECT dari project httparchive dataset \textit{technologies} tabel 2020\_08\_01\_* dengan menggunakan perintah FROM. Kolom akan diurutkan berdasarkan url secara \textit{ascending}. Kolom akan dibatasi sebanyak 10 baris dengan menggunakan perintah LIMIT. Sepuluh contoh hasil keluaran dari \textit{query} diatas dapat dilihat pada tabel \ref{table:contoh_langkah2}:

\begin{table}[H]
\centering
\begin{tabular}{|l|l|l|}
	\hline
	\textbf{Row} & \textbf{url} & \textbf{app}\\
	\hline
	1 & http://0-1.ru/ & Liveinternet\\
	\hline
	2 & http://0-1.ru/ & Yandex.Metrika\\
	\hline
	3 & http://0-1.ru/ & IIS\\
	\hline
	4 & http://0-1.ru/ & Microsoft ASP.NET\\
	\hline
	5 & http://0-1.ru/ & YouTube\\
	\hline
	6 & http://0-1.ru/ & Windows Server\\
	\hline
	7 & 	
	http://0-10-10.cocolog-nifty.com/  & 	
	Nginx \\
	\hline
	8 & 	
	http://0-10-10.cocolog-nifty.com/  & Twitter\\
	\hline
	9 & 	
	http://0-10-10.cocolog-nifty.com/  & jQuery\\
	\hline
	10 & 	
	http://0-10-10.cocolog-nifty.com/  & Osano\\
	\hline
\end{tabular}
\caption{Contoh Aplikasi Yang Digunakan Website}
\label{table:contoh_langkah2}
\end{table}

\subsection{Mengelompokkan Berdasarkan Nama Semua Aplikasi Yang Dipakai}
Pengelompokan aplikasi dapat dilakukan dengan menggunakan \textit{query}. Berikut adalah \textit{query} yang digunakan.
\begin{lstlisting}
SELECT tabelName.app, num.num_sites , versioned.versioned_count , unversioned.unversioned_count
FROM 
(SELECT DISTINCT app
FROM `httparchive.technologies.2020_08_01_*` ) tabelName

LEFT JOIN 

(SELECT tabel1.app, count(app) AS versioned_count
FROM `httparchive.technologies.2020_08_01_*` AS tabel1
WHERE tabel1.app!="" AND tabel1.info != "" 
GROUP BY tabel1.app) AS versioned

ON(versioned.app = tabelName.app)

LEFT JOIN

(SELECT tabel2.app, count(app) AS unversioned_count
FROM `httparchive.technologies.2020_08_01_*` AS tabel2
WHERE tabel2.app!="" AND tabel2.info = "" 
GROUP BY tabel2.app) AS unversioned

ON (unversioned.app = tabelName.app)

LEFT JOIN 

(SELECT app, count(url) AS num_sites
FROM `httparchive.technologies.2020_08_01_*`
GROUP BY app) AS num

ON (tabelName.app = num.app)
LIMIT 10
\end{lstlisting}

Pada \textit{query} diatas akan dibuat beberapa tabel baru yang bersifat sementara. Pada tabel tersebut akan dilakukan pemilihan pada kolom app dengan menggunakan perintah SELECT dan menggunakan DISTINCT agar app yang ditampilkan hanya keluar satu kali. Data diambil dari project httparchive dataset \textit{technologies} tabel 2020\_08\_01\_* dengan menggunakan perintah FROM. Kemudian tabel akan digabungkan dengan tabel lain. Kolom lain berisikan jumlah aplikasi yang memiliki versi, jumlah aplikasi yang tidak memiliki versi, dan jumlah situs yang menggunakan aplikasi tertentu. Kemudian dengan menggunakan perintah SELECT, akan dipanggil beberapa variabel dari setiap kolom dari setiap tabel. Kolom yang diambil berupa: app, jumlah situs yang dipakai aplikasi (num\_sites), jumlah aplikasi yang memiliki versi (versioned\_count), dan jumlah aplikasi yang tidak memiliki versi (unversioned\_count). Kolom akan dibatasi sebanyak 10 baris dengan menggunakan perintah LIMIT. Sepuluh contoh hasil keluaran dari \textit{query} diatas dapat dilihat pada tabel \ref{table:contoh_langkah3}:

\begin{table}[H]
\centering
\begin{tabular}{|l|l|r|r|r|}
	\hline
	\textbf{Row} & \textbf{app} & \textbf{num\_sites} & \textbf{versioned\_count} & \textbf{unversioned\_count}\\
	\hline
	1 & jQuery & 10.003.030 & 9.979.001 & 24.029\\
	\hline
	2 & Apache & 4.067.380 & 1.118.200 & 2.949.180\\
	\hline
	3 & PHP & 5.977.790 & 2.522.620 & 3.455.170\\
	\hline
	4 & MySQL & 4.047.343 & null & 4.047.343\\
	\hline
	5 & Microsoft SharePoint & 14.419 & 11.402 & 3.017\\
	\hline
	6 & YouTube & 1.028.360& null & 1.028.360\\
	\hline
	7 & Microsoft ASP.NET & 865.276 & 407.366 & 457.910\\
	\hline
	8 & Google Code Prettify & 32.171 & null & 32.171\\
	\hline
	9 & Typekit & 253.890 & 253.203 & 687\\
	\hline
	10 & Slick & 759.805 & 66.249 & 693.556\\
	\hline
\end{tabular}
\caption{Hasil Pengelompokan Aplikasi Beserta Jumlah \textit{Versioned} Dan \textit{Unversioned}}
\label{table:contoh_langkah3}
\end{table}

Pada \cite{pascal}, jumlah data yang digunakan lebih sedikit sehingga jumlah keseluruhan data juga akan berbeda. Terdapat beberapa aplikasi yang sama sehingga dapat dibandingkan datanya. Tabel pada \cite{pascal} dapat dilihat pada tabel \ref{table:contoh_tabel_paper}: 

\begin{table}[H]
\centering
\begin{tabular}{|p{1cm}|wr{0.1\linewidth}|wr{0.11\linewidth}|wr{0.13\linewidth}|wr{0.1\linewidth}|p{3 cm}|wr{0.12\linewidth}|}
	\hline
	\textbf{Name} & \textbf{num-sites} & \multirow{2}{0.11\linewidth}{\textbf{avg-confidence}} & \multirow{2}{0.13\linewidth}{\textbf{num-unversioned}} & \multirow{2}{0.1\linewidth}{\textbf{num-versioned}} & \textbf{website} & \multirow{3}{0.12\linewidth}{\textbf{num-supported-version}}\\
	&  &  &  &  &  &  \\
	&  &  &  &  &  &  \\
	\hline
	jQuery & 1.011 & 99.70 & 14 & 997 & https://jquery.com & >=3 \\
	\hline
	Boot strap & 340 & 99.30 & 88 & 342 & https://getboot strap.com & >=4 \\
	\hline
	JQuery Migrate & 298 & 99.66 & 31 & 267 & https://github.com /jquery/jquery-migrate & ?\\
	\hline
	PHP & 591 & 99.83 & 348 & 245 & https://www.php .net & >=7.2 \\
	\hline
	Font Awesome & 400 & 99.50 & 160 & 240 & https://fontaweso me.com & >=5 \\
	\hline
	JQuery UI & 176 & 99.43 & 7 & 169 & https://jqueryui .com & ? \\
	\hline
	Word Press & 346 & 100.00 & 181 & 165 & https://wordpress .org & >=5.4.2 \\
	\hline
	Under score.js & 124 & 24.19 & 2 & 122 & https://underscore js.org & ? \\
	\hline
	Lodash & 125 & 59.20 & 3 & 122 & https://lodash.com & ? \\
	\hline
	
	
\end{tabular}
\caption{Tabel Sepuluh Data Aplikasi Pada \cite{pascal}}
\label{table:contoh_tabel_paper}
\end{table}

\subsection{Mencari Data Tentang Versi Aplikasi Yang Masih Didukung}
Sebelum menentukan suatau aplikasi usang atau tidak, kita harus mencari versi dari setiap aplikasi secara manual. Versi setiap aplikasi dapat dilihat di-\textit{official documentation} dari setiap aplikasi. Hasil pencarian dari aplikasi yang masih didukung dapat dilihat pada tabel \ref{lamp:A}. 

\subsection{Melakukan Perbandingan Antara Versi Aplikasi Yang Masih Dipakai Sekarang Dengan Versi Aplikasi Yang Masih Didukung}
Setelah mendapatkan data versi minimal dari setiap aplikasi, data tersebut akan dibandingkan dengan versi aplikasi yang dipakai \textit{url}. \textit{Supported} adalah versi aplikasi dari yang dipakai \textit{url} masih mendukung atau diatas atau sama dengan versi yang didukung didokumen. \textit{unsupported} adalah versi aplikasi dari yang dipakai url sudah tidak mendukung atau dibawah versi yang didukung didokumen. \textit{not\_versioned} adalah versi aplikasi dari url tidak ditampilkan. \textit{non\_conclusive} adalah versi aplikasi tidak dapat ditentukan. 
\begin{lstlisting}
	CREATE TEMP FUNCTION normaizedSemanticVersion(semanticVersion STRING) 
	AS ((
	SELECT STRING_AGG(
	IF(isDigit, REPEAT('0', 100 - LENGTH(chars)) || chars, chars) ORDER BY grp 
	)
	FROM (
	SELECT grp, isDigit, STRING_AGG(char, '' ORDER BY OFFSET) chars,
	FROM (
	SELECT OFFSET, char, isDigit,
	COUNTIF(NOT isDigit) OVER(ORDER BY OFFSET) AS grp
	FROM UNNEST(SPLIT(semanticVersion, '')) AS char WITH OFFSET, 
	UNNEST([char IN ('1','2','3','4','5','6','7','8','9','0')]) isDigit
	)
	GROUP BY grp, isDigit
	)));
	CREATE TEMP FUNCTION compareSemanticVersions(
	normSemanticVersion1 STRING, 
	normSemanticVersion2 STRING) 
	AS ((
	SELECT CASE 
	WHEN info < min_supported THEN 'UNSUPPORTED'
	ELSE 'SUPPORTED'
	END
	FROM UNNEST([STRUCT(
	normaizedSemanticVersion(normSemanticVersion1) AS info, 
	normaizedSemanticVersion(normSemanticVersion2) AS min_supported
	)])
	));
	WITH test AS (
	SELECT url, app, info, min_supported	
	FROM `httparchive-bigquery-346414.Step.app_min_supported_and_info`
	)
	SELECT url, app, info, min_supported, if(info = '', "NOT VERSIONED",if(min_supported = '?','NON CONCLUSIVE',compareSemanticVersions(info, min_supported)) ) as  result
	FROM test 
	ORDER BY url
\end{lstlisting}


Berikut ini adalah hasil sepuluh data yang dapat dilihat pada tabel \ref{table:contoh_langkah5.1}. Data diambil berdasarkan banyak aplikasi yang dipakai oleh url tertentu. 
\begin{table}[H]
\centering
\begin{tabular}{|l|r|r|r|r|}
	\hline
	\textbf{url} & \textbf{supported} & \textbf{unsupported} & \textbf{not\_versioned} & \textbf{non\_conclusive}\\
	\hline
	authservice.pegipegi.com & 0 & 9 & 224 & 2\\
	\hline
	serviceauth.pegipegi.com & 0 & 13 & 220 & 2\\
	\hline
	mcatselfprep.com &0 & 14 & 52 & 8\\
	\hline
	perpetua.it & 0 & 14 & 50 & 12\\
	\hline
	sulava.com & 0 & 10 & 59 & 10\\
	\hline
	
	theraceclub.com & 2 & 12 & 48 & 16\\
	\hline
	
	jobs.discover.com & 4 & 8 & 58 & 8\\
	\hline
	
	dickssportinggoods.jobs & 4 & 8 & 56 & 8 \\
	\hline
	careers.symphonytalent.com & 4 & 8 & 56 & 8 \\
	\hline
	
	jobs.cedarfair.com & 4 & 8 & 52 & 12\\
	\hline
\end{tabular}
\caption{Hasil Perbandingan Aplikasi Berdasarkan url}
\label{table:contoh_langkah5.1}
\end{table}

Data juga dikelompokkan berdasarkan \textit{category} yang memiliki aplikasi yang \textit{unsupported}. Kemudian \textit{category} tersebut dihitung kembali dan dikelompokkan berdasarkan jumlah aplikasi yang \textit{unsupport}-nya. Hasil dapat dilihat pada tabel \ref{table:contoh_langkah5.2}.
\begin{table}[H]
\centering
\begin{tabular}{|r|r|r|r|r|}
	\hline
	\textbf{n=0} & \textbf{n=1} & \textbf{n=2} & \textbf{n=3} & \textbf{n>=4}\\
	\hline
	2 & 1 & 0 & 0 & 58\\
	\hline
\end{tabular}
\caption{Jumlah Category Dengan Aplikasi Unsupported}
\label{table:contoh_langkah5.2}
\end{table}

Data juga dibandingkan berdasarkan aplikasi tertentu. Data yang dihasilkan adalah num\_sites atau jumlah url yang menggunakan aplikasi tertentu, app, supported atau aplikasi yang masih didukung, unsupported atau aplikasi yang sudah tidak didukung, not\_versioned atau aplikasi yang tidak diberi informasi versi, dan non\_conclusive atau versi aplikasi tidak dapat ditentukan. Hasil dari data dapat dilihat pada tabel \ref{table:contoh_langkah5.3}.
\begin{adjustwidth}{-2.5 cm}{-2.5 cm}\centering\begin{threeparttable}[!htb]
	\begin{tabular}{|r|l|r|r|r|r|}
		\hline
		\textbf{num\_sites} & \textbf{app} & \textbf{supported} & \textbf{unsupported} & \textbf{not\_versioned} & \textbf{non\_conclusive}\\
		\hline
		10.003.030 &jQuery &1.604.830 &8.374.171 &24.029 &0 \\
		\hline
		8.190.668 &Google Analytics &0 &0 &8.190.668 &0 \\
		\hline
		7.494.642 &WordPress &350 &4.891.016 &2.603.276 &0 \\
		\hline
		7.230.612 &Nginx &652 &1.789.692 &5.440.268 &0 \\
		\hline
		5.977.790 &PHP &167.095 &2.355.525 &3.455.170 &0 \\
		\hline
		5.481.111 &Google Font API &0 &0 &5.481.111 &0 \\
		\hline
		4.529.823 &Google Tag Manager &0 &0 &4.529.823 &0 \\
		\hline
		4.067.380 &Apache &764.690 &353.510 &2.949.180 &0 \\
		\hline
		4.047.343 &MySQL &0 &0 &4.047.343 &0 \\
		\hline
	\end{tabular}
	\caption{Hasil Perbandingan Aplikasi}
	\label{table:contoh_langkah5.3}
\end{threeparttable}\end{adjustwidth}


\section{Hasil Sample Data Dengan Beberapa Aplikasi}
Diambil satu data sample dengan aplikasi dan nomor versinya. Data sample tersebut merupakan data Apache . Data dapat dilihat pada gambar \ref{fig:data_sample_res}.
\begin{figure}[H]
\centering  
\includegraphics[scale=0.4]{Gambar/apache.PNG}  
\caption{Data Sample Jumlah Aplikasi Dengan Versi yang Dipakai} 
\label{fig:data_sample_res} 
\end{figure}
Pada data \ref{fig:data_sample_res} terdapat bagian bawah yang menunjukkan informasi versi dari aplikasi dan bagian kiri merupakan jumlah url yang menggunakan aplikasi. Chart yang berwarna merah adalah chart yang menunjukkan versi aplikasi tersebut sudah tidak didukung. Chart yang berwarna biru menunjukkan versi aplikasi terseubt masih didukung. Sedangkan warna hijau menandakan versi aplikasi tidak dapat ditentukan.