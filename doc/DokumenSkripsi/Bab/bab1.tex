%versi 2 (8-10-2016) 
\chapter{Pendahuluan}
\label{chap:intro}
Pada bab ini dijelaskan latar belakang, rumusan masalah, tujuan, batasan masalah, metodologi, dan sistematika pembahasan.
\section{Latar Belakang}
\label{sec:label}
Di masa teknologi saat ini, banyak perusahaan yang menggunakan \textit{website} sebagai tempat untuk mencari informasi. Terdapat banyak \textit{website} yang menggunakan aplikasi yang sudah usang. HTTPArchive memiliki dataset penggunaan aplikasi setiap \textit{website} dari \textit{desktop} dan \textit{mobile} pada bulan Januari tahun 2016 sampai sekarang yang dapat dilihat menggunakan teknologi BigQuery. Dataset pada HTTPArchive didapatkan dari \textit{Chrome User Experience Report} (CrUX). CrUX merupakan \textit{public dataset} dari \textit{user experience} data pada jutaan \textit{website}. Pengujian pada \textit{dataset} tersebut dilakukan dengan menggunakan \textit{Chrome} pada \textit{desktop} dan \textit{mobile}.  

HTTPArchive \footnote{https://github.com/HTTPArchive/httparchive.org/blob/main/docs/gettingstarted$\_$bigquery.md} adalah sebuah \textit{project} yang bersifat \textit{open source} untuk melihat bagaimana \textit{website} dibuat. Di dalam HTTPArchive terdapat data-data historis yang disediakan untuk menunjukkan bagaimana \textit{website} terus berkembang dan \textit{project} ini sering digunakan untuk penelitian. Didalam HTTPArchive terdapat \textit{dataset} yang berisi jutaan \textit{website} setiap bulan dan dapat dianalisis menggunakan teknologi BigQuery. 

BigQuery \cite{bqIntroduction} adalah salah satu produk dari Google untuk menyimpan kumpulan data yang berbasis \textit{cloud} dan dapat digunakan untuk menganalisis data. BigQuery dapat menjalankan \textit{query} dalam skala \textit{terabyte} dalam hitungan detik dan \textit{petabyte} dalam hitungan menit. BigQuery digunakan untuk menganalisis data yang besar dengan melakukan \textit{query}.  

Pada \cite{pascal} dilakukan penelitian tentang seberapa besar penggunaan aplikasi usang pada \textit{website} di Indonesia. Data diambil dari \textit{website} Alexa, dari 1.500 situs teratas menurut peringkat Alexa untuk pengunjung situs di Indonesia dan mengidentifikasi jenisnya aplikasi yang dipakai beserta nomor versinya, lebih dari setengah atau 63\% aplikasi yang digunakan berhasil dibandingkan dengan skrip yang telah dibuat dan hasilnya aplikasi tidak lagi didukung oleh pengelolanya.

Beberapa aplikasi sudah menyediakan fitur untuk memperbarui ke versi yang paling baru tanpa harus menginstal ulang. Dalam kebanyakan kasus, versi aplikasi yang semakin baru sudah memperbaiki banyak kerentanan yang sudah diketahui. Beberapa aplikasi usang tidak memiliki pemberitahuan untuk meng-\textit{update} sehingga pengguna tidak mengetahui jika terdapat \textit{update}. Aplikasi yang baik biasanya memberikan \textit{update} otomatis dan memberikan pesan yang efektif jika terjadi \textit{update}. 

Pada skripsi ini, dibuat sebuah replikasi dari \cite{pascal} tetapi dengan data yang lebih besar. Data dapat diambil dari HTTPArchive dengan melakukan \textit{query} pada BigQuery. Pada penelitian ini dilakukan perhitungan pada jumlah aplikasi yang sudah diberi versi dan belum diberi versi. Terdapat beberapa aplikasi yang informasi versinya tidak dapat ditentukan. Versi aplikasi yang tidak dapat ditentukan disebut NON-CONCLUSIVE. Versi aplikasi yang NON-CONCLUSIVE biasanya berisi simbol. Selain itu terdapat beberapa versi aplikasi yang kosong. Versi aplikasi yang kosong disebut sebagai NOT-VERSIONED Versi aplikasi yang dipakai setiap \textit{website} juga dibandingkan dengan versi aplikasi yang masih didukung berdasarkan \textit{official website}-nya. Pada penelitian ini hanya berfokus pada \textit{semantic version}. \textit{Semantic version} merupakan tata cara penentuan urutan pada aplikasi. Pada \textit{semantic version} terdapat tiga komponen pengurutannya, yaitu \textit{major}, \textit{minor}, dan \textit{patch}. Kemudian hasil tersebut ditampilkan dalam bentuk \textit{bar chart}.


\section{Rumusan Masalah}
\label{sec:rumusan}
Berikut ini adalah rumusan masalah dari penelitian ini:
\begin{enumerate}
	\item Bagaimana cara mengambil nilai informasi untuk mendukung pencarian aplikasi yang usang?
	\item Bagaimana mereplikasi jurnal \cite{pascal} dengan menggunakan data yang lebih besar?
	\item Berapa banyak \textit{website} pada HTTPArchive yang menggunakan aplikasi yang semua aplikasinya masih didukung?
\end{enumerate}


\section{Tujuan}
\label{sec:tujuan}
Berikut ini adalah tujuan dari penelitian ini:
\begin{enumerate}
	\item Mengambil nilai informasi dengan cara melakukan query untuk mengumpulkan daftar website, mencari aplikasi yang digunakan, mencari aplikasi yang digunakan website, mengelompokkan berdasar nama semua aplikasi yang dipakai, mencari data tentang versi aplikasi yang masih didukung, dan melakukan perbandingan antara versi aplikasi yang masih dipakai sekarang dengan aplikasi yang masih didukung.
	\item Mereplikasi jurnal \cite{pascal} dengan menggunakan data yang lebih besar, dikarenakan jurnal \cite{pascal} menggunakan data yang lebih sedikit, sehingga \textit{chart} dari aplikasi yang ditampilkan juga lebih sedikit. Data pada jurnal \cite{pascal} ruang lungkupnya hanya sebatas wilayah Indonesia saja, sedangkan pada skripsi ini dilakukan penelitian dengan ruang lingkup global atau dunia. 
	\item Mencari jumlah \textit{website} pada HTTPArchive yang menggunakan aplikasi yang semua aplikasinya masih didukung.
\end{enumerate}


\section{Batasan Masalah}
\label{sec:batasan}
Berikut ini adalah batasan masalah dari penelitian ini:
\begin{enumerate}
    \item Data HTTPArchive yang digunakan adalah data \textit{technologies} pada bulan Agustus tahun 2020.
    \item Versi aplikasi berisi simbol merupakan NON-CONCLUSIVE.
    \item Pada penelitian ini hanya berfokus pada \textit{semantic version} yang dimana digit \textit{major}, \textit{minor}, dan \textit{patch} adalah angka.
    \item \textit{Chart} pada aplikasi ini tidak menunjukkan data yang \textit{UNVERSIONED} dan \textit{NON-CONCLUSIVE} karena datanya terlalu besar sehingga mengakibatkan data lain tidak terlihat.
\end{enumerate}

\section{Metodologi}
\label{sec:metlit}
Bagian-bagian pekerjaan skripsi ini adalah sebagai berikut:
\begin{enumerate}
	\item Melakukan studi literatur mengenai HTTPArchive.
	\item Melakukan studi literatur mengenai Query
	\item Melakukan studi literatur mengenai BigQuery.
	\item Melakukan studi literatur mengenai ReactJS.
	\item Melakukan studi literatur mengenai NodeJS.
	\item Melakukan studi literatur mengenai ChartJS.
	\item Melakukan studi literatur mengenai \textit{website} yang dikatakan usang.
	\item Menganalisis beberapa \textit{website} yang dikatakan usang.
	\item Membandingkan versi aplikasi pada data HTTPArchive yang dipakai sekarang dengan versi aplikasi yang ada pada dokumentasi.
	\item Membuat perangkat lunak untuk menampilkan data.
	\item Menulis dokumen skripsi.
\end{enumerate}


\section{Sistematika Pembahasan}
\label{sec:sispem}
Laporan penelitian tersusun ke dalam enam bab secara sistematis sebagai berikut.
\begin{itemize}
    \item Bab 1 Pendahuluan\\
    Berisi latar belakang, rumusan masalah, tujuan, batasan masalah, metodologi penelitian, dan sistematika pembahasan.
   
   \item Bab 2 Dasar Teori\\
    Berisi teori BigQuery, teori HTTPArchive, teori Node.js, teori React.js, Chart.js, dan Express.js.
   
    \item Bab 3 Percobaan Awal\\
    Berisi eksplorasi BigQuery, dataset yang digunakan di HTTPArchive, langkah \textit{query} yang dilakukan dengan data terbatas untuk membandingkan versi aplikasi, dan hasil \textit{sample} data Apache.
  
    \item Bab 4 Penggalian Data\\
    Berisi langkah-langkah query yang dilakukan untuk mengumpulan data yang besar, penjelasan tentang sample data antara aplikasi apache dan nginx, php dan python, dan jquery dan jquery migrate.
    
    \item Bab 5 Pembangunan Perangkat Lunak\\
Berisi perancangan perangkat lunak yang dibangun, masukan dan keluaran dari perangkat lunak.
   
    \item Bab 6 Kesimpulan dan Saran\\
    Berisi kesimpulan dari awal hingga akhir penelitian dan saran untuk penelitian berikutnya.
    \end{itemize}
